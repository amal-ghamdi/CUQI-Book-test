\documentclass[11pt,a4paper]{article}

%% Use the option review to obtain double line spacing
%% \documentclass[preprint,review,12pt]{elsarticle}

%% Use the options 1p,twocolumn; 3p; 3p,twocolumn; 5p; or 5p,twocolumn
%% for a journal layout:
%% \documentclass[final,1p,times]{elsarticle}
%% \documentclass[final,1p,times,twocolumn]{elsarticle}
%% \documentclass[final,3p,times]{elsarticle}
%% \documentclass[final,3p,times,twocolumn]{elsarticle}
%% \documentclass[final,5p,times]{elsarticle}
%% \documentclass[final,5p,times,twocolumn]{elsarticle}

%% The graphicx package provides the includegraphics command.
\usepackage[utf8]{inputenc}
\usepackage[T1]{fontenc}
\usepackage[spanish,english]{babel}
\usepackage{color}
\usepackage{xcolor}
\usepackage{graphicx}
\usepackage{amsmath,amssymb,amsfonts,amsthm}
\usepackage{dsfont,bm,bbm} 
\usepackage{mathtools}
\mathtoolsset{showonlyrefs,showmanualtags}
\usepackage{colortbl}
\usepackage{subfig}
\usepackage{todonotes} 
\usepackage{multirow}
\usepackage[top=1in, left=1in, right=1in, bottom=1in]{geometry}
\usepackage{csquotes}
\usepackage{textcomp}
\usepackage{nicefrac}
\usepackage{url}
\usepackage[firstinits=true,style=numeric,backend=bibtex,maxbibnames=6]{biblatex}
\usepackage[colorlinks=true,linkcolor=blue,breaklinks=true,pdfauthor={FU},]{hyperref}
\usepackage{listings}
\decimalpoint

%===NEWCOMMANDS=========================================
\newcommand{\given}{\;\ifnum\currentgrouptype=16 \middle\fi|\;}
\newcommand{\suchthat}{\;\ifnum\currentgrouptype=16 \middle\fi|\;}
\newcommand{\norm}[1]{\left\Vert#1\right\Vert}
\newcommand{\abs}[1]{\left\vert#1\right\vert}
\newcommand{\iprod}[1]{\left\langle#1\right\rangle}
\newcommand{\BibLaTeX}{{Bib}\LaTeX~}
\newcommand{\dd}{\mathrm{d}}
\newcommand{\tran}{\mathsf{T}}
\newcommand{\python}{$\text{python}^{\text{\texttrademark}}$~}
\newcommand{\pr}[1]{\mathds{P}\!\left[#1\right]}
\newcommand{\pf}{p_\mathcal{F}}
\newcommand{\pfe}{\widehat{p}_\mathcal{F}}
\newcommand{\like}[1]{\mathrm{L}\!\left(#1\right)}
\newcommand{\loglike}[1]{\ln\mathrm{L}\!\left(#1\right)}
\newcommand{\I}[2]{{\mathds{1}_{#1}\!\left(#2\right)}}
\newcommand{\E}[2]{\mathds{E}_{#1}\!\left[{#2}\right]}
\newcommand{\V}[2]{\mathds{V}_{#1}\!\left[{#2}\right]}
\newcommand{\ve}[1]{\bm{#1}}
\newcommand{\mat}[1]{\mathbf{#1}}
\newcommand{\cov}[1]{\mathds{C}\mathrm{ov}\left[#1\right]}
\newcommand{\cv}[1]{\widehat{\mathrm{cv}}\left(#1\right)} 
\newcommand{\cve}{\widehat{\mathrm{cv}}} 
\newcommand{\deter}{\mathrm{det}}
\newcommand*{\bfrac}[2]{\genfrac{}{}{0pt}{}{#1}{#2}}
\newcommand{\Dkl}[2]{{D}_{\text{KL}}\left(#1||#2\right)}
\newcommand{\Dklb}[2]{D_{\text{KL}}(#1||#2)}
\renewcommand{\labelitemi}{\tiny$\blacksquare$}
\DeclareMathOperator*{\argmax}{arg\,max}
\DeclareMathOperator*{\argmin}{arg\,min}
\DeclareMathOperator{\diag}{diag}
\DeclareMathOperator{\supp}{supp} % "essential support"
\DeclareMathOperator{\esssupp}{ess\,supp} % "essential support"


\begin{document}

% \begin{frontmatter}
%% Title, Authors and addresses
\title{Description of the target applications}
\author{02975 Introduction to uncertainty quantification for inverse problems}
\date{}
\maketitle


Data for inference of each target application are provided in the folder \texttt{homework\_data}. This file contains the variables: \texttt{dim} (dimension of the field discretization), \texttt{endpoint} (length of the physical domain), \texttt{dim\_obs} (number of data points), \texttt{true\_field} (underlying true field for comparison to your posterior solution), \texttt{sigma\_obs} (noise standard deviation, $\sigma_{\rm obs}$), \texttt{y\_exact} (forward response field generated by the truth), \texttt{y\_obs} (data points), \texttt{grid\_obs} (sensor locations).


\section{Poisson problem}
Consider a heat conductive rod of length $L = \pi$ with a varying conductivity (the conductivity of the rod changes from point to point). We fix the temperature at the end-points of the rod and apply a heat source distributed along the length of the rod. We wait until the rod reaches an equilibrium temperature distribution. The equilibrium temperature of the rod is modelled using the second order steady-state PDE as

\begin{equation}\label{eq:diff_eq_1D}
\left\{
\begin{aligned}
& \dfrac{\dd}{\dd x}\left(X(x,\omega) \dfrac{\dd Y(x,\omega)}{\dd x}\right) = -f(x), \quad & x\in (0,L) \\
& Y(0,\omega) = Y(\pi,\omega) = 0.
\end{aligned}
\right.
\end{equation}
Here, $Y$ represents the temperature distribution along the rod, $X(x, \omega) $ is the unkown conductivity of the rod and $f(x)$ is a constant and deterministic heat source given by
\begin{equation*}
	f(x) = 10\exp( -\frac{ (x - \pi/2)^2} {0.02} ).
\end{equation*}
To ensure that the conductivity of the rod is non-negative, we assume that $X$ is a $\log$ of another field, i.e. $X( \cdot , \omega ) = \exp( U( \cdot , \omega ) )$ where $U$ is not necessarily positive.

This problem is implemented in the CUQI toolbox as $$\texttt{model}=\texttt{cuqi.testproblem.Poisson\_1D}(N, L, f, \texttt{field\_type}, \texttt{KL\_map}),$$
where $N$ is the number of discretization points for $U$, $L=\pi$ is the length of the domain, $\texttt{source}=f(x) $ is a function with the source term implementation, $\texttt{field\_type}=\{\texttt{None},\texttt{KL},\texttt{KL\_Full},\allowbreak\texttt{CustomKL},\texttt{Step}\}$ defines the random field generation (specified in the projects), and $\texttt{KL\_map}$ is a function with the transformation of a Gaussian field (in our case is always $x=\exp(u)$). To obtain a forward solution on the domain $D$ given a Gaussian parameter vector, we simply use $\texttt{y~=~model.forward(u)}$




\section{Deconvolution problem}
The mathematical model for convolution of a random signal on a one-dimensional spatial domain $D=[0,1]$, can be written as a stochastic Fredholm integral equation of the first kind:
\begin{equation}\label{eq:1D_deconv}
Y(x,\omega) = \int_{0}^{1} k(x,x')X(x',\omega)\,\dd x' \quad\text{with}\quad k(x,x') = \frac{b}{2\exp\left(-b\abs{x-x'}\right)},
\end{equation}
where $Y(x,\omega)$ denotes the convolved signal random variable and we assume a deterministic convolution kernel $k$ with fixed parameter $b=48$. In practice, a finite-dimensional representation of \eqref{eq:1D_deconv} is employed. After discretizing the signal domain into $N$ components, the convolution model can be expressed as a system of linear algebraic equations $\ve{y}(\omega)=\mat{K}\ve{x}(\omega)$. 

This problem is implemented in the CUQI toolbox as $$\texttt{model}=\texttt{cuqi.model.Deconv\_1D}(N, L, \texttt{kernel}, \texttt{field\_type}, \texttt{KL\_map}),$$
where $N$ is the domain discretization, $L=1$ is the length of the domain, $\texttt{kernel}$ is a function with the convolution kernel implementation $\texttt{field\_type}=\{\texttt{None},\texttt{KL},\texttt{CustomKL},\texttt{Step}\}$ defines the random field generation (specified in the projects), and $\texttt{KL\_map}$ is a function with the transformation of a Gaussian field (in our case is always $x=\exp(u)$). To obtain a forward solution on the domain $D$ given a Gaussian parameter vector, we simply use $\texttt{y~=~model.forward(u)}$

\end{document}